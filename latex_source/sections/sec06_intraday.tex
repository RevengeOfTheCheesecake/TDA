% ==============================================================================
% SECTION 6: INTRADAY DATA ANALYSIS (Condensed)
% ==============================================================================

\section{Intraday Data Analysis}
\label{sec:intraday}

\subsection{Motivation}

The 60-day rolling windows in Section~\ref{sec:methodology} yield noisy correlation estimates. With standard error $SE(\hat{\rho}) \approx (1-\rho^2)/\sqrt{n} \approx 0.11$ for $n=60$ and $\rho=0.4$, correlations carry $\pm 0.22$ uncertainty at 95\% confidence. This noise propagates into topological features.

\textbf{Hypothesis}: Intraday data (5-minute bars) will reduce correlation estimation variance, producing more stable topology while preserving genuine market structure.

\subsection{Methodology}

\textbf{Data}: 5-minute bars for 20 stocks (January 2023--December 2024) via Alpha Vantage API. After preprocessing: 39,876 observations across 511 trading days.

\textbf{Topology computation}: Rolling correlation windows of 780 bars ($\approx$ 60 trading days), sampled daily for comparison with Section~\ref{sec:methodology}.

\subsection{Results}

\subsubsection{Stability Improvement}

\begin{table}[H]
\centering
\caption{Topological Feature Stability: Daily vs Intraday}
\label{tab:intraday-stability}
\begin{tabular}{@{}lcccc@{}}
\toprule
\textbf{Frequency} & \textbf{Mean $H_1$} & \textbf{Std Dev} & \textbf{CV} & \textbf{Improvement} \\
\midrule
Daily & 4.23 & 2.87 & 0.678 & -- \\
Intraday & 4.19 & 1.92 & 0.458 & \textbf{$-32.4\%$} \\
\bottomrule
\end{tabular}
\end{table}

\textbf{Key finding}: Mean $H_1$ loop count remains nearly identical (4.23 vs 4.19), while standard deviation decreases 33\%. The CV reduction of 32.4\% confirms that intraday data filters sampling noise while preserving genuine topological structure.

Statistical tests: $t$-test for means ($p = 0.76$, no difference) and Levene's test for variances ($p < 0.001$, significant reduction).

\subsubsection{Crisis Detection}

\begin{table}[H]
\centering
\caption{Crisis Detection Performance (VIX $> 30$ days)}
\label{tab:crisis-detection}
\begin{tabular}{@{}lcccc@{}}
\toprule
\textbf{Estimator} & \textbf{TPR} & \textbf{FPR} & \textbf{AUC} \\
\midrule
Daily & 0.68 & 0.32 & 0.72 \\
Intraday & 0.77 & 0.19 & \textbf{0.81} \\
\bottomrule
\end{tabular}
\end{table}

The 9-point AUC improvement (0.72 $\to$ 0.81, $p = 0.003$) comes primarily from reduced false positives ($-41\%$), important for practical risk management.

\subsubsection{Trading Performance}

\begin{table}[H]
\centering
\caption{Strategy Performance with Intraday Topology}
\label{tab:strategy-intraday}
\begin{tabular}{@{}lccc@{}}
\toprule
\textbf{Configuration} & \textbf{Sharpe} & \textbf{CAGR} & \textbf{Max DD} \\
\midrule
Daily topology & $-0.56$ & $-13.55\%$ & $-34.68\%$ \\
Intraday topology & $-0.41$ & $-10.22\%$ & $-28.94\%$ \\
\textbf{Improvement} & +27\% & +25\% & +17\% \\
\bottomrule
\end{tabular}
\end{table}

Performance improves but remains negative (Sharpe $-0.41$), confirming that improved estimation alone cannot overcome the fundamental design limitations identified in Section~\ref{sec:analysis}.

\subsection{Implications}

\textbf{Sample size guidance}: For CV $< 0.45$ (acceptable stability), practitioners need either 12+ years of daily data or 2+ years of intraday data for 20-asset universes.

\textbf{Methodological contribution}: The preservation of mean $H_1$ (4.23 vs 4.19) across frequencies validates that persistent homology detects genuine market structure, not sampling artifacts.

\textbf{Limitation}: The Epps effect (Epps, 1979) causes correlations to decline at high frequencies due to non-synchronous trading. Applying the Hayashi-Yoshida (2005) estimator would likely further reduce topology noise.

\textbf{Conclusion}: Intraday data improves topology stability by 32\% and crisis detection by 9 AUC points, but cannot overcome the cross-sector heterogeneity problem. The fundamental issue is correlation structure, not sample size---motivating the sector-specific approach in Section~\ref{sec:sector}.

